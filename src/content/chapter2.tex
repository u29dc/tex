\section{Chapter 2}

\subsection{Subsection 1}

\lipsum[10]

\begin{verbatim}
>> syms x, int(1/(1-x^4))
ans =
atan(x)/2 + atanh(x)/2
\end{verbatim}

\lipsum[20]

\begin{figure}[ht]
    \centering
    \includegraphics[width=0.5\textwidth]{example1}
    \caption{A half text width image}
    \label{fig:halfwidthimage}
\end{figure}

\begin{quote}
  Hasselmo, et al.\ (1995) investigated\dots
\end{quote}

See Figure \ref{fig:blackbox}. Here is how you add footnotes.\footnote{Sample footnote}

\lipsum[2]

\begin{center}
  \url{https://example.com}
\end{center}

\begin{figure}[ht]
  \centering
  \fbox{\rule[-0.5cm]{4cm}{4cm} \rule[-0.5cm]{4cm}{0cm}}
  \caption{A black box}
  \label{fig:blackbox}
\end{figure}

\lipsum[9]

\lipsum[2]

\begin{lstlisting}[language=Python]
import numpy as np

def incmatrix(genl1,genl2):
    m = len(genl1)
    n = len(genl2)
    M = None #to become the incidence matrix
    VT = np.zeros((n*m,1), int)  #dummy variable
    
    #compute the bitwise xor matrix
    M1 = bitxormatrix(genl1)
    M2 = np.triu(bitxormatrix(genl2),1) 

    for i in range(m-1):
        for j in range(i+1, m):
            [r,c] = np.where(M2 == M1[i,j])
            for k in range(len(r)):
                VT[(i)*n + r[k]] = 1;
                VT[(i)*n + c[k]] = 1;
                VT[(j)*n + r[k]] = 1;
                VT[(j)*n + c[k]] = 1;
                
                if M is None:
                    M = np.copy(VT)
                else:
                    M = np.concatenate((M, VT), 1)
                
                VT = np.zeros((n*m,1), int)
    return M
\end{lstlisting}
